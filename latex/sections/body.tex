% TODO: Write a proper introduction.

% Thesis
The Aeschylus's Homeric\footnote{Set within the broader narrative of the Iliad
and the Odyessy} play \emph{Agamemnon} is set in a world where the status of
women is consistently placed beneath that of men. Despite this, the character of
Clytemnestra consistently confronts, and challenges the inequitable gender-norms
set forth in the play Agamemnon. Her \emph{"man-like"} behavior is noted by the
townsmen of Argos, and eventually culminates to a high point, in her slaying of
her husband Agamemnon. The killing of Agamemnon becomes the definite moment
where Clytemnestra ceases to conform within the norms of womanhood, and acts in
the stead of a male hero, seeking justice through revenge.

\clearpage % Prevents orphan sentence

% First define what we mean 'gender norms', lest we commit an anachronism.
How does Clytemnestra challenge the gender norms in \emph{Agamemnon}? In order
to begin answering this question, we must first understand what the gender norms
are, as well as what gender norms are displayed in the play. A gender norm is a
set of behaviours and attitudes considered appropriate or acceptable for a given
gender. Such norms naturally differ throughout history and culture, and it
would be anachronistic for modern-day gender-norms to be used as a baseline
of comparison. Likewise, it would be fallacious for one to speak about the
gender norms of "Ancient Greece" --- for as Herodotus can attest, the cultures
of the Mediterranean are diverse and heterogenous. This is why we make the
explicit distinction of \emph{gender norms set forth in Agamemnon}, as opposed
to the prior two cases. This way, any claims about gender norms in the play
\emph{Agamemnon} can be substantiated through textual evidence.

% This is where we enumerate the instances where gender norms are displayed.
Therefore, let us begin with an examination of the gender norms in Aeschylus's
play \emph{Agamemnon}. From the very beginning of the play, it is hinted to us
that the role of a woman is secondary to that of a man, even if said woman is a
Queen in her own right. When the Argive townsmen\footnote{This essay
will use Argive townsmen to refer to the various differing titles used in the
Loebs and Fagles translations of \emph{Agamemnon}. Argive townsmen can stand for
either Chorus (as used in the Loeb edition) or Leader, Chorus, Elder, as used in
the Fagles.} arrive to greet Clytemnestra, they address her with:

\begin{quote}
    We've come, Clytemnestra. We respect your power. \\
    Right is it to honour the warlord's woman, \\
    \textit{once he leaves the throne}. \\
    \autocite[258]{fagles}
\end{quote}

\noindent
It is clear that the Argive townsmen address her with honour, but yet --- it is
also apparent that such honour is conditional on the basis that their King is
at sea. Likewise, when Clytemnestra shares the news of Troy's sacking, the
Argive townsmen respond with skepticism:

\begin{quote}
  \textsc{Leader}: \\
  And you have proof?

  \textsc{Clytemnestra}: \\
  I do, I must. Unless the god is lying.

  \textsc{Leader}: \\
  That, or a phantom spirit sends you into raptures.

  \textsc{Clytemnestra}: \\
  No one takes me in with visions --- senseless dreams.

  \textsc{Leader}: \\
  Or giddy rumour, you haven't indulged yourself ---

  \textsc{Clytemnestra}: \\
  You treat me like a child, you mock me?

  \autocite[275]{fagles}
\end{quote}

\noindent
One may argue that the Argive townsmen are displaying merely the due skepticism
associated with any extraordinary news, but that claim is difficult to make in
light of tone of reproach which is used. The fagles translation of the passage
is tempered by the fact that it looks like a dialogue, a mutual discussion
between Clytemnestra and the Argive townsmen. In the original Greek, all 3
passages from the Chorus are interrogatives, to which the Loeb edition of the
text renders as:

\begin{quote}
  \textsc{Chorus}: \\
  What then is the proof? Hast thou warrenty of this?

  \textsc{Clytemnestra}: \\
  I have, indeed; unless some god hath played me false.

  \textsc{Chorus}: \\
  Dost thou pay regard to the persuasive visions of dreams?

  \textsc{Clytemnestra}: \\
  I would not heed the fancies of a slumbering brain.

  \textsc{Chorus}: \\
  But can it be some pleasing rumour that hath fed thy hopes?

  \textsc{Clytemnestra}: \\
  Truly thou floutest mine understanding as it were a child's?

  \autocite[275]{loeb}
\end{quote}

\noindent
The conversation seems less akin to an discussion between equals, but clearly
takes on an condescending attitude --- as if Clytemnestra is somehow less
capable then they are. And one must keep in mind that the townsmen are
not merely addressing anyone, but a Queen in her own right. This attitude of
disrespect towards Clytemnestra does not end here, but continues on even after
she explains her system of beacons and message fires. The Argive townsmen gather
amongst themselves, and discuss Clytemnestra's address:

\begin{quote}
  --- \emph{Just like a woman} \\
  to fill with thanks before the truth is clear.

  --- So gullible. Their stories spread like wildfire, \\
  they fly fast and die faster; \\
  \emph{rumours voiced by women come to nothing.}

  \autocite[474]{fagles}
\end{quote}

\noindent
From this passage, it is clear that their dismissive attitudes towards
Clytemnestra's news is not founded upon any critique of how her news is
delivered, nor even on the basis that she is the Queen. But rather, because she
is a woman. This display of contempt for a woman's speech is duly contrasted in
an earlier segment, where the Argive townsmen remark that Clytemnestra's
speech is \emph{"man-like"}, for it is excellent and well spoken:

\begin{quote}
  \textsc{Leader}: \\
  Spoken like a man, my lady, loyal, \\
  full of self-command. I've heard your sign \\
  and now your vision. \\
  \ldots\

  \autocite[355]{fagles}
\end{quote}

% Analysis of what these quotes come together to mean for our thesis.
Now that we have these examples displayed in front of us, what are the
conclusions that we can draw? It is clear that the within the play
\emph{Agamemnon}, the status of a woman is consistently placed beneath that of a
man. For clearly, Clytemnestra's word is doubted at every turn, for the reason
that she is a woman, as opposed to anything else. We see in lines 474 that a
woman's voice is literally said to be worth nothing. The way that the Argive
townsmen question Clytemnestra at lines 275 is condescending, and strongly
implies that the men do not think much of her abilities, even though
Clytemnestra is a Queen. Likewise, we see that qualities such as "self-command"
or eloquence in speech are associated with men rather than woman, hence the
accolade of "man-like" which was bestowed upon to Clytemnestra. Hence, in
summary - we can say that the play \emph{Agamemnon} is set in a world where
women are seen as secondary, inferior, more gullible, and less capable.

% Second part of the thesis --- on how Clytemnestra challenges said norms.
With this in mind, we may move on to the second leg of our thesis. In what ways
does Clytemnestra challenge these gender norms? From the very beginning, we see
Clytemnestra object to the condescending questions of the Argive townsmen. She
replies to their words with scorn, asking: \emph{"You treat me like a child,
you mock me?"} \autocite[275]{fagles}. She clearly realises that her abilities
are doubted, and does not allow their rudeness to go unchallenged. Furthermore,
when the Argive townsmen send the Herald to tell Clytemnestra the news of Troy's
ssacking, she chides them for now believing in her:

\begin{quote}
  \textsc{Clytemnestra}: \\
  I cried out long ago! ---

  for joy, when the first herald came burning

  through the night and told the city's fall.

  And there were some who smiled and said,

  `A few fires persuade you Troy's in ashes.

  Women, women, elated over nothing.'

  You made me seem deranged.

  For all that I sacrificed --- a woman's way,

  you'll say --- station to station on the walls

  \ldots\

  \autocite[580]{fagles}
\end{quote}

\noindent
In the passage above, Clytemnestra is shown to be keenly aware of the
patronising comments of the Argive townsmen, and challenges their preconceptions
with the proof that they are wrong. This way, she acknowledges the existence of
these gender norms, and confronts them rather than let them pass. In the Loeb
translation of this passage, the Herald remarks in response that \emph{"Boast
like to this, laden to the full with truth, misbeseems not the speech of a
nobel wife"} \autocite[613]{loeb}.

\noindent
It is significant here that another person within the play of \emph{Agamemnon}
was able to note how Clytemnestra is acting outside the boundaries of what is
normative for her gender. This occures again, when Clytemnestra greets
Agamemnon with the purple tapestries to step on. During the course of their
argument, Agamemnon incredulously responds to Clytemnestra's urging, with:
\emph{"Surely 'tis not a woman's part to be fond of contest."}
\autocite[940]{loeb}
