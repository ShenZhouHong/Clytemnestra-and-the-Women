In Aeschylus's \textit{Agamemnon}, the character of Clytemnestra consistently
confronts the gender-norms of Homeric literature. Her man-like behaviour is
noted by the townsmen of Argos, and eventually culminates in the slaying of
her husband Agamemnon, on his return back from Troy. Clytemnestra's
remorseless speech and audacious actions is contrasted with the meek
deliberations of the Argive townsmen, who falter at the screams of their king.
This essay seeks to examine how Clytemnestra's pursuit of justice requires a
subversion of her role as a woman in the norms of Homeric literature, and how
by righting the sacrifice of Iphigenia through the mete of blood, Clytemnestra
acts in the role of the (male) Homeric hero instead.

In order to begin an examination of Clytemnestra, and how her actions challenge
the gender-norms of Homeric literature, it is important to have an clear
understanding of what these norms are in the first place. A gender norm is a
set of behaviours and attitudes considered appropriate or acceptable for a given
gender. Such norms naturally differ throughout history and culture, and it
would be anachronistic for modern-day gender-norms to be used as a baseline
of comparison. Likewise, it would be fallacious for one to speak about the
gender norms of "Ancient Greece" --- for as Herodotus can attest, the cultures
of the Mediterranean are diverse and heterogenous. This is why, the distinction
of "Homeric literature" is made. For \textit{Agamemnon} is a play set within the
framing context of Homer's \textit{Iliad} and {Odyssey} --- stories from the
proverbial `Age of Heroes'. Therefore, by examining \textit{Agamemnon}, as well
as the \textit{Iliad} and the \textit{Odyessy}, we can establish a baseline of
what these norms are --- as well as how Clytemnestra breaks them in
\textit{Agamemnon}.

The role of women is secondary to that of men, in the gender-norms of Homeric
literature. Passivity and deference are the norm, and outside of the rancour
of the Gods, we see no examples of female assertiveness, in the world of Homer.
Beginning with the \textit{Iliad}, the `Rage of Achilles' opens
with the ransoming of Chryseis, and the subsequent confiscation of Briseis from
Achilles. The treatment of Briseis and Chryseis shows that (at least within the
scope of the Trojan War,) women are often little more than objects --- captives,
or the spoils of war. One may argue that the demeanour of Briseis and Chryseis
is explained by the fact that they are slaves, but even Helen --- a queen in her
own right, acts in relative passivity. Most telling is the scene where she
protests at the cowardince of Paris, but yields without a word to his reply:

\begin{quote}
    "... \\
    Wait, \\
    take my advice and call a halt right here: \\
    no more battling with fiery-haired Menelaus, \\
    pitting strength against strength in single combat---\\
    madness. \emph{He} just might impale you on his spear!" \\
    \\
    But Paris replied at once to Helens
\end{quote}

Likewise, in the \textit{Agamemnon}, we see a similar portrayal of the inferior
role of women. Most noteworthy is the time when the Chorus, the Argive townsmen
pay their visit to Clytemnestra, and say:

\begin{quote}
    We've come, Clytemnestra. We respect your power. \\
    Right is it to honour the warlord's woman, \\
    \textit{once he leaves the throne}. \\
    \autocite[258]{fagles}
\end{quote}

\noindent Furthermore, during the interval when the townsmen of Argos wait for
the arrival of the Herald, we are treated with a display of contempt and
incredulity at the word of Clytemnestra. And it is clearly said, that these
sentiments are attributed to the fact that she is a woman, as opposed to the
novelty of her system of fire relays, or anything else:

\begin{quote}
    --- Just like a woman \\
    to fill with thanks before the truth is clear. \\
    --- So guillible. Their stories spread like wildfire, they fly fast and die faster; \\
    rumours voiced by women come to nothing. \\
    \autocite[470]{fagles}
\end{quote}

What do these examples show us?
