In Aeschylus's \textit{Agamemnon}, the character of Clytemnestra consistently
confronts the gender-norms of Homeric literature. Her man-like behaviour is
noted by the townsmen of Argos, and eventually culminates in the slaying of
her husband Agamemnon, on his return back from Troy. Clytemnestra's
remorseless speech and audacious actions is contrasted with the meek
deliberations of the Argive townsmen, who falter at the screams of their king.
This essay seeks to examine how Clytemnestra's pursuit of justice requires a
subversion of her role as a woman in the norms of Homeric literature, and how
by righting the sacrifice of Iphigenia through the mete of blood, Clytemnestra
acts in the role of the (male) Homeric hero instead.

In order to begin an examination of Clytemnestra, and how her actions challenge
the gender-norms of Homeric literature, it is important to have an clear
understanding of what these norms are in the first place. A gender norm is a
set of behaviours and attitudes considered appropriate or acceptable for a given
gender. Such norms naturally differ throughout history and culture, and it
would be anachronistic for modern-day gender-norms to be used as a baseline
of comparison. Likewise, it would be fallacious for one to speak about the
gender norms of "Ancient Greece" --- for as Herodotus can attest, the cultures
of the Mediterranean are diverse and heterogenous. This is why, the distinction
of "Homeric literature" is made. For \textit{Agamemnon} is a play set within the
framing context of Homer's \textit{Iliad} and {Odyssey} --- stories from the
proverbial `Age of Heroes'. Therefore, by examining \textit{Agamemnon}, as well
as the \textit{Iliad} and the \textit{Odyessy}, we can establish a baseline of
what these norms are --- as well as how Clytemnestra breaks them in
\textit{Agamemnon}.
