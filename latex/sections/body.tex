% TODO: Write a proper introduction.

% Thesis
Aeschylus' Homeric\footnote{Set within the broader narrative of the Iliad
and the Odyssey} play \emph{Agamemnon} is set in a world where the status of
women is placed beneath that of men. Despite this, the character of
Clytemnestra consistently confronts, and challenges the inequitable gender-norms
set forth in the play \emph{Agamemnon}. Her \emph{"man-like"} behavior is noted
by the townsmen of Argos, and eventually culminates to a high point, in her
slaying of her husband Agamemnon. The killing of Agamemnon becomes the definite
moment where Clytemnestra ceases to conform within the norms of womanhood, and
acts in the stead of a male hero, seeking justice through revenge.

% First define what we mean 'gender norms', lest we commit an anachronism.
How does Clytemnestra challenge the gender norms in \emph{Agamemnon}? In order
to begin answering this question, we must first understand what the gender norms
are, as well as what gender norms are displayed in the play. A gender norm is a
set of behaviors, characteristics, and attitudes considered "normal" or
appropriate for a given gender. Such norms naturally differ throughout history
and culture, and it would be anachronistic for modern-day gender-norms to be
used as the baseline of comparison. Likewise, it would be fallacious for one to
speak about the gender norms of "Ancient Greece" --- for as Herodotus can
attest, the cultures of the Mediterranean are diverse and heterogenous. This is
why we make the explicit distinction of \emph{gender norms set forth in
Agamemnon}, as opposed to the prior two cases. Hence any claims about
gender norms in the play \emph{Agamemnon} can be substantiated through textual
evidence.

% This is where we enumerate the instances where gender norms are displayed.
Therefore, let us begin with an examination of the gender norms in Aeschylus's
play \emph{Agamemnon}. From the very beginning of the play, it is hinted to us
that the role of a woman is secondary to that of a man, even if said woman is a
Queen in her own right. When the Argive townsmen\footnote{This essay
will use Argive townsmen to refer to the various differing titles used in the
Loeb and Fagles translations of \emph{Agamemnon}. Argive townsmen can stand for
either Chorus (as used in the Loeb edition) or Leader, Chorus, Elder, as used in
the Fagles.} arrive to greet Clytemnestra, they address her with:

\begin{quote}
    We've come, Clytemnestra. We respect your power. \\
    Right is it to honor the warlord's woman, \\
    \textit{once he leaves the throne}\footnote{Emphasis my own}.

    \autocite[258]{fagles}
\end{quote}

\noindent
It is clear that the Argive townsmen address her with honor, but yet --- it is
also apparent that such honor is conditional on the basis that their King is
at sea. Likewise, when Clytemnestra shares the news of Troy's sacking, the
Argive townsmen respond with skepticism:

\begin{quote}
  \textsc{Leader}: \\
  And you have proof?

  \textsc{Clytemnestra}: \\
  I do, I must. Unless the god is lying.

  \textsc{Leader}: \\
  That, or a phantom spirit sends you into raptures.

  \textsc{Clytemnestra}: \\
  No one takes me in with visions --- senseless dreams.

  \textsc{Leader}: \\
  Or giddy rumor, you haven't indulged yourself ---

  \textsc{Clytemnestra}: \\
  You treat me like a child, you mock me?

  \autocite[275]{fagles}
\end{quote}

\noindent
One may argue that the Argive townsmen are displaying merely the due skepticism
associated with any extraordinary news, but that claim is difficult to make in
light of tone of reproach which is used. The Fagles translation of the passage
is tempered by the fact that it looks like a dialogue: a mutual discussion
between Clytemnestra and the Argive townsmen. In the original Greek, all 3
passages from the Chorus are interrogatives, to which the Loeb edition of the
text renders as:

\begin{quote}
  \textsc{Chorus}: \\
  What then is the proof? Hast thou warrenty of this?

  \textsc{Clytemnestra}: \\
  I have, indeed; unless some god hath played me false.

  \textsc{Chorus}: \\
  Dost thou pay regard to the persuasive visions of dreams?

  \textsc{Clytemnestra}: \\
  I would not heed the fancies of a slumbering brain.

  \textsc{Chorus}: \\
  But can it be some pleasing rumour that hath fed thy hopes?

  \textsc{Clytemnestra}: \\
  Truly thou floutest mine understanding as it were a child's?

  \autocite[275]{loeb}
\end{quote}

\noindent
The conversation seems less akin to an discussion between equals, but clearly
takes on an condescending attitude --- as if Clytemnestra is somehow less
capable then they are. And one must keep in mind that the townsmen are
not merely addressing anyone, but a Queen in her own right. This attitude of
disrespect towards Clytemnestra does not end here, but continues on even after
she explains her system of beacons and message fires. The Argive townsmen gather
amongst themselves, and discuss Clytemnestra's address:

\begin{quote}
  --- \emph{Just like a woman}\footnote{Emphasis my own} \\
  to fill with thanks before the truth is clear.

  --- So gullible. Their stories spread like wildfire, \\
  they fly fast and die faster; \\
  \emph{rumours voiced by women come to nothing.}\footnotemark[4]

  \autocite[474]{fagles}
\end{quote}

\noindent
From this passage, it is clear that their dismissive attitudes towards
Clytemnestra's news is not founded upon any critique of her system of beacons,
nor even on the basis of the kind of person she is. But rather, because she
is a woman. This display of contempt for a woman's speech is duly contrasted in
an earlier segment, where the Argive townsmen remark that Clytemnestra's
speech is \emph{"man-like"}, for it is excellent and well spoken:

\begin{quote}
  \textsc{Leader}: \\
  Spoken like a man, my lady, loyal, \\
  full of self-command. I've heard your sign \\
  and now your vision.

  \autocite[355]{fagles}
\end{quote}

% Analysis of what these quotes come together to mean for our thesis.
Now that we have these examples displayed in front of us, what are the
conclusions that we can draw? It is clear that the within the play
\emph{Agamemnon}, the status of a woman is consistently placed beneath that of a
man. For clearly, Clytemnestra's word is doubted at every turn, for the reason
that she is a woman, as opposed to anything else. We see in lines 474 that a
woman's voice is literally said to be worth nothing. The way that the Argive
townsmen question Clytemnestra at lines 275 is condescending, and strongly
implies that the men do not think much of her abilities, even though
Clytemnestra is a Queen. Likewise, we see that qualities such as "self-command"
or eloquence in speech are associated with men rather than woman, hence the
accolade of "man-like" which was bestowed upon to Clytemnestra. Hence, in
summary --- we can say that the play \emph{Agamemnon} is set in a world where
women are seen as secondary, inferior, more gullible, and less capable.

% Second part of the thesis --- on how Clytemnestra challenges said norms.
With this in mind, we may move on to the second portion of our thesis. In what
ways does Clytemnestra challenge these gender norms? From the very beginning,
we see Clytemnestra object to the condescending questions of the Argive
townsmen. She replies to their words with scorn, asking: \emph{"You treat me
like a child, you mock me?"} \autocite[275]{fagles}. She clearly realizes that
her abilities are doubted, and does not allow their rudeness to go
unchallenged. Furthermore, when the Argive townsmen send the Herald to tell
Clytemnestra the news of Troy's sacking, she chides them for now believing in
her:

\begin{quote}
  \textsc{Clytemnestra}: \\
  I cried out long ago! --- \\
  for joy, when the first herald came burning \\
  through the night and told the city's fall. \\
  And there were some who smiled and said, \\
  `A few fires persuade you Troy's in ashes. \\
  Women, women, elated over nothing.' \\
  You made me seem deranged. \\
  For all that I sacrificed --- a woman's way, \\
  you'll say --- station to station on the walls

  \autocite[580]{fagles}
\end{quote}

\noindent
Clytemnestra is shown to be keenly aware of the
patronizing comments of the Argive townsmen, and challenges their preconceptions
with the proof that they are wrong. This way, she acknowledges the existence of
these gender norms, and confronts them rather than let them pass. In the Loeb
translation of this passage, the Herald remarks in response that to \emph{"Boast
like to this, laden to the full with truth, misbeseems not the speech of a
nobel wife"} \autocite[613]{loeb}.

\noindent
It is significant here that another person within the play of \emph{Agamemnon}
was able to note how Clytemnestra is acting outside the boundaries of what is
normative for her gender. This is not only a singular occurrence, but happens
multiple times in the play. Her speech is called "man-like" by the Herald.
Likewise, when Clytemnestra greets Agamemnon with the request to step on the
purple tapestries, Agamemnon incredulously responds to Clytemnestra's urging
with: \emph{"Surely 'tis not a woman's part to be fond of contest."}
\autocite[940]{loeb}. Both of these examples serve to show that Clytemnestra's
non-conformance with what's expected of a woman is noted not only by ourselves,
but also by other characters. This is the surest indication of how she acts
outside of what is expected of a woman, for it depends solely on the text
itself.

% Final leg of the thesis, about the slaying of Agamemnon
However, ultimately up till now we have gone through only the more superficial
examples of how Clytemnestra challenges the gender norms of \emph{Agamemnon}.
The climax of the play lays at the slaying of Agamemnon at the hands of
Clytemnestra, and it is ultimately this murder which sets Clytemnestra apart
from the women. For the murder of Agamemnon was not merely any killing, but an
enactment of justice  in revenge for the death of Iphigenia. Already, the
murder of a king at the hands of a woman is unimaginable. In fact, the
prophecies of Cassandra automatically makes the Argive townsmen imagine a man:

\begin{quote}
  \textsc{Cassandra}: \\
  You pray, and they close in to kill!

  \textsc{Leader}: \\
  What \emph{man}\footnote{Emphasis my own} prepares this, this dreadful ---

  \textsc{Cassandra}: \\
  Man? \\
  You \emph{are}\footnotemark[5] lost, to every word I've said.

  \autocite[1262]{fagles}
\end{quote}

\noindent
For not only is violence a task that is exclusively relegated to the domain of
men, but the very act of killing for vengeance --- the sort of justice that is
only paid through a blood price --- is reserved solely by men as well. In both
the Odyssey and the Iliad, men are the ones who venture forth to right wrongs
by the means of the sword. The case of Clytemnestra killing Agamemnon for the
sake of Iphigenia has parallels to Odysseus's murder of suitors. For in both
cases, we have the parent seeking vengeance for the child. Clytemnestra's
murder of Agamemnon is the act that truly propels her outside the boundaries of
what's normative for her gender, and she takes the role of a male, Homeric hero
--- by seeking justice in her hands. And indeed, when the murder happens, the
reader is treated to a scene of emasculation --- as the Argive townsmen meekly
squabble amongst themselves, at the sound of their lord's death. Indeed, the
word "witless" comes to mind as a good description of the Argive townsmen's
inaction.  Only when Clytemnestra reveals the body of Agamemnon, do the Argives
react:

\begin{quote}
  \textsc{Leader}:\\
  You appal me, you, your brazen words --- \\
  exulting over your fallen king.

  \textsc{Clytemnestra}: \\
  And you, \\
  you try me like \emph{some desperate woman.} \\
  My heart is steel, well you know. Praise me, \\
  blame me as you choose. It's all one. \\
  Here is Agamemnon, my husband made a corpse \\
  by this right hand --- a masterpiece of Justice. \\
  Done is done.

  \autocite[1425]{fagles}
\end{quote}

\noindent
Of particular note is Clytemnestra's line: \emph{"And you, try me like some
desperate woman."} \autocite[1426]{fagles}, to which the Loeb edition renders as
\emph{"Ye are proving me as if I were a witless woman."} \autocite[1426]{loeb}.
One can interpret this passage as Clytemnestra expressing scorn or distain at
how the Argives yet again think of her as unintelligent and incapable. However,
taking into account of the next line: "My heart is steel" --- it sounds more
like an rejection of her role within womanhood. The contempt expressed in these
words read as if Clytemnestra is saying: "stop thinking of me as a \emph{mere}
woman, but rather as Clytemnestra. For I have done the deeds that you see
before you, and I am capable of them." And finally, when the Argive townsmen
threaten Clytemnestra with the punishment of exile, her rueful reply makes it
clear both the reason she murdered Agamemnon, as well as her thoughts on the
double-standards in play:

\begin{quote}
  \textsc{Clytaemnestra}:\\
  And now you sentence me? ---\\
  you banish \emph{me} from the city, curses breathing \\
  down my neck? But \emph{he} ---\\
  name one charge you brought against him then,\\
  He thought no more of it than killing a beast,\\
  and his flocks were rich, teeming in their fleece,\\
  but he sacrificed his own child, our daughter,\\
  the agony I laboured into love\\
  to charm away the savage winds of Thrace.\footnote{Original emphasis from
  Fagles}

  \autocite[1437]{fagles}
\end{quote}

\noindent
The fact that Clytemnestra killed Agamemnon out of a love for her own child is
a powerful motif that we must not allow to escape our notice. In her words, she
clearly laments the sheer disregard of Iphigenia that was shown: \emph{"He
thought no more of it than killing a beast"} \autocite[1440]{fagles}. Knowing
already that the world of \emph{Agamemnon} is a world where even the highest of
women are consistently treated as inferior to men, it is not a difficult
conjecture to think that the sacrifice of Iphigenia is made easier by the
fact that she is a woman. We cannot possibly know whether or not Clytemnestra
had these thoughts in mind, but through her actions --- by
repaying Iphigenia's death with Agamemnon's life, Clytemnestra makes an explicit
declaration: That the life of her child is \emph{not} that of a mere head of
cattle, but that Iphigenia's life is just as worthy to be paid in the life of a
king. This is why Agamemnon's slaying is the high point of Clytemnestra's
refusal to conform to the gender norms of women --- for both in action as well
as in symbol, she affirms: that not only she is just as capable, that she is
also able to pursue justice, but that the worthiness of a woman's life is of
equal standing to that of a man's. She speaks deeply of the injustice behind
the killing of her daughter, and draws attention to the bitter irony of how
Agamemnon is able to commit murder without punishment, but she is held to an
literal double-standard.

% Begin conclusion
In conclusion, Clytemnestra consistently confronts and challenges the gender
norms set forth in Agamemnon. By analyzing the text, we're able to establish
what these norms are, as well as see how Clytemnestra breaks them. The contempt
and condescension of the Argive townsmen makes it clear that women are seen as
inferior to men --- with words such as "witless" used to describe woman. And
despite this, Clytemnestra is able to defy these norms, to the notice of those
around her. Ultimately, the climax of the play is also where Clytemnestra
transcends the norms of her gender --- for the killing of Agamemnon is a
symbolic moment where Clytemnestra takes upon the role of a male hero,
seeking justice through revenge. Not only that, the murder of Agamemnon for the
murder of Iphigenia is an symbolic affirmation that the life of Iphigenia is
worth just as much as the life of the King.

\noindent
